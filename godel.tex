\documentclass{article}
\usepackage[utf8]{inputenc}

\title{Gödel's incompleteness theorems}
\author{Gaurav Kar}
\date{July 2021}

\begin{document}

\maketitle

\section{Summary}
Gödel's incompleteness theorems are two theorems of mathematical logic that are concerned with the limits of provability in formal axiomatic theories. These results, published by Kurt Gödel in 1931, are important both in mathematical logic and in the philosophy of mathematics. The theorems are widely, but not universally, interpreted as showing that Hilbert's program to find a complete and consistent set of axioms for all mathematics is impossible.

The first incompleteness theorem states that no consistent system of axioms whose theorems can be listed by an effective procedure (i.e., an algorithm) is capable of proving all truths about the arithmetic of natural numbers. For any such consistent formal system, there will always be statements about natural numbers that are true, but that are unprovable within the system. The second incompleteness theorem, an extension of the first, shows that the system cannot demonstrate its own consistency.
Gödel's second incompleteness theorem shows that, under general assumptions, this canonical consistency statement Cons(F) will not be provable in F. The theorem first appeared as "Theorem XI" in Gödel's 1931 paper "On Formally Undecidable Propositions in Principia Mathematica and Related Systems I". In the following statement, the term "formalized system" also includes an assumption that F is effectively axiomatized. Second Incompleteness Theorem: "Assume F is a consistent formalized system which contains elementary arithmetic. Then {\displaystyle F\not \vdash {\text{Cons}}(F)}{\displaystyle F\not \vdash {\text{Cons}}(F)}.

This theorem is stronger than the first incompleteness theorem because the statement constructed in the first incompleteness theorem does not directly express the consistency of the system. The proof of the second incompleteness theorem is obtained by formalizing the proof of the first incompleteness theorem within the system F itself.
His incompleteness theorems meant there can be no mathematical theory of everything, no unification of what’s provable and what’s true. What mathematicians can prove depends on their starting assumptions, not on any fundamental ground truth from which all answers spring.

In the 89 years since Gödel’s discovery, mathematicians have stumbled upon just the kinds of unanswerable questions his theorems foretold. For example, Gödel himself helped establish that the continuum hypothesis, which concerns the sizes of infinity, is undecidable, as is the halting problem, which asks whether a computer program fed with a random input will run forever or eventually halt. Undecidable questions have even arisen in physics, suggesting that Gödelian incompleteness afflicts not just math, but — in some ill-understood way — reality.
\end{document}
