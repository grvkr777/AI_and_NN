\documentclass{article}
\usepackage[utf8]{inputenc}
\usepackage[a4paper, total={6in, 8.5in}]{geometry}

\title{Moravec's Paradox}
\author{Gaurav Kar (18111025)}
\date{July 2021}

\begin{document}

\maketitle

\section{Summary}
Moravec's paradox is the observation by artificial intelligence and robotics researchers that, contrary to traditional assumptions, reasoning requires very little computation, but sensorimotor skills require enormous computational resources.The principle was articulated by Hans Moravec, Rodney Brooks, Marvin Minsky and others in the 1980s. One possible explanation of the paradox, offered by Moravec, is based on evolution. All human skills are implemented biologically, using machinery designed by the process of natural selection. In the course of their evolution, natural selection has tended to preserve design improvements and optimizations. The older a skill is, the more time natural selection has had to improve the design. Abstract thought developed only very recently, and consequently, we should not expect its implementation to be particularly efficient.
A compact way to express this argument would be:
\begin{itemize}
    \item We should expect the difficulty of reverse-engineering any human skill to be roughly proportional to the amount of time that skill has been evolving in animals.
\end{itemize}
\begin{itemize}
       \item The oldest human skills are largely unconscious and so appear to us to be effortless.
\end{itemize}
\begin{itemize}
    \item Therefore, we should expect skills that appear effortless to be difficult to reverse-engineer, but skills that require effort may not necessarily be difficult to engineer at all.
\end{itemize}
Some examples of skills that have been evolving for millions of years: recognizing a face, moving around in space, judging people's motivations, catching a ball, recognizing a voice, setting appropriate goals, paying attention to things that are interesting; anything to do with perception, attention, visualization, motor skills, social skills and so on.
Some examples of skills that have appeared more recently: mathematics, engineering, games, logic and scientific reasoning. These are hard for us because they are not what our bodies and brains were primarily evolved to do. These are skills and techniques that were acquired recently, in historical time, and have had at most a few thousand years to be refined, mostly by cultural evolution.

In the early days of artificial intelligence research, leading researchers often predicted that they would be able to create thinking machines in just a few decades (see history of artificial intelligence). Their optimism stemmed in part from the fact that they had been successful at writing programs that used logic, solved algebra and geometry problems and played games like checkers and chess. Logic and algebra are difficult for people and are considered a sign of intelligence. Many prominent researchers assumed that, having (almost) solved the "hard" problems, the "easy" problems of vision and commonsense reasoning would soon fall into place. They were wrong, and one reason is that these problems are not easy at all, but incredibly difficult. The fact that they had solved problems like logic and algebra was irrelevant, because these problems are extremely easy for machines to solve.
\end{document}
