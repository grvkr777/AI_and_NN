\documentclass{article}
\usepackage[utf8]{inputenc}
\usepackage[a4paper, total={5.9in, 7in}]{geometry}

\title{Flat Earth Theory and AI}
\author{Gaurav Kar (18111025)}
\date{July 2021}

\begin{document}

\maketitle

\section{Summary}
The flat Earth model is an archaic conception of Earth's shape as a plane or disk. Many ancient cultures subscribed to a flat Earth cosmography, including Greece until the classical period (323 BC), the Bronze Age and Iron Age civilizations of the Near East until the Hellenistic period (31 BC), India until the Gupta period (early centuries AD), and China until the 17th century. The idea of a spherical Earth appeared in ancient Greek philosophy with Pythagoras (6th century BC), although most pre-Socratics (6th–5th century BC) retained the flat Earth model. In the early 4th century BC Plato wrote about a spherical Earth, and by about 330 BC his former student, Aristotle, had provided strong empirical evidence for this. Knowledge of the Earth's global shape then gradually began to spread beyond the Hellenistic world. Despite the scientific fact of Earth's sphericity, pseudoscientific flat Earth conspiracy theories are espoused by modern flat Earth societies and, increasingly, by unaffiliated individuals using social media.

In the Internet era, the proliferation of communications technology and social-media platforms such as YouTube, Facebook and Twitter have given individuals, famous or otherwise, a platform to spread pseudo-scientific ideas and build stronger followings. The flat-Earth conjecture has flourished in this environment. Social media and the internet, furthermore, have made it easier for like-minded theorists to connect with one another and mutually reinforce their beliefs. In other words, social media has had a "levelling effect", in that experts have less sway in the public mind than they used to. YouTube had faced criticism for allowing the spread of misinformation and conspiracy theories through its platform. In 2019, YouTube stated that it was making changes in its software to reduce the distribution of videos based on conspiracy theories including flat Earth.

Organizations skeptical of fringe beliefs have occasionally performed tests to demonstrate the local curvature of the Earth. One of these, conducted by members of the Independent Investigations Group at the Salton Sea on June 10, 2018, was attended also by supporters of a flat Earth, and the encounter between the two groups was recorded by the National Geographic Explorer. This experiment successfully demonstrated the curvature of the Earth by the disappearance over distance of boat-based and shore-based targets.
Flat earth theory has been debunked nth number of times by using simple scientific demonstrations, using AI only further cements the claim that earth is spherical and not flat, so AI can be used to debunk the claim that earth is flat and AI cannot bring this topic from "pseudoscience" to "science" 

\end{document}
